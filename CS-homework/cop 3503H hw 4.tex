
\documentclass[10pt]{article}

\usepackage[margin=.75in]{geometry} 
\usepackage{amsmath,amsthm,amssymb}
\usepackage[linesnumbered,ruled]{algorithm2e}
\DontPrintSemicolon  

\newcommand{\R}{\mathbb{R}}  
\newcommand{\Z}{\mathbb{Z}}
\newcommand{\N}{\mathbb{N}}
\newcommand{\Q}{\mathbb{Q}}
\newcommand{\C}{\mathbb{C}}

\usepackage{mathtools}

\usepackage[T1]{fontenc}
\usepackage{titling}
\usepackage{listings}
\usepackage{color}
\usepackage{enumitem}

\definecolor{dkgreen}{rgb}{0,0.6,0}
\definecolor{gray}{rgb}{0.5,0.5,0.5}
\definecolor{mauve}{rgb}{0.58,0,0.82}

\lstset{frame=tb,
  language=Java,
  aboveskip=2mm,
  belowskip=2mm,
  showstringspaces=false,
  columns=flexible,
  basicstyle={\small\ttfamily},
  numbers=none,
  numberstyle=\tiny\color{gray},
  keywordstyle=\color{blue},
  commentstyle=\color{dkgreen},
  stringstyle=\color{mauve},
  breaklines=true,
  breakatwhitespace=true,
  tabsize=3,
}

\DeclarePairedDelimiter{\ceil}{\lceil}{\rceil}
%gets rid of qed symbol
\renewcommand{\qedsymbol}{}

\newenvironment{theorem}[2][Theorem]{\begin{trivlist}
		\item[\hskip \labelsep {\bfseries #1}\hskip \labelsep {\bfseries #2.}]}{\end{trivlist}}
\newenvironment{lemma}[2][Lemma]{\begin{trivlist}
		\item[\hskip \labelsep {\bfseries #1}\hskip \labelsep {\bfseries #2.}]}{\end{trivlist}}
\newenvironment{exercise}[2][Exercise]{\begin{trivlist}
		\item[\hskip \labelsep {\bfseries #1}\hskip \labelsep {\bfseries #2.}]}{\end{trivlist}}
\newenvironment{problem}[2][Problem]{\begin{trivlist}
		\item[\hskip \labelsep {\bfseries #1}\hskip \labelsep {\bfseries #2.}]}{\end{trivlist}}
\newenvironment{question}[2][Question]{\begin{trivlist}
		\item[\hskip \labelsep {\bfseries #1}\hskip \labelsep {\bfseries #2.}]}{\end{trivlist}}
\newenvironment{corollary}[2][Corollary]{\begin{trivlist}
		\item[\hskip \labelsep {\bfseries #1}\hskip \labelsep {\bfseries #2.}]}{\end{trivlist}}

\newenvironment{solution}{\begin{proof}[Solution]}{\end{proof}}

\begin{document}
	
	
	
	
	% ------------------------------------------ %
	%                 START HERE                 %
	% ------------------------------------------ %
	\setlength{\topmargin}{0in}
	
	\setlength{\droptitle}{-13em} 
	\title{Homework 4} % Replace X with the appropriate number
	\author{Evan Waldmann\\COP 3503H} % Replace "Author's Name" with your name
	
	\maketitle
	
	% -----------------------------------------------------
	% The following two environments (theorem, proof) are
	% where you will enter the statement and proof of your
	% first problem for this assignment.
	%
	% In the theorem environment, you can replace the word
	% "theorem" in the \begin and \end commands with
	% "exercise", "problem", "lemma", etc., depending on
	% what you are submitting. Replace the "x.yz" with the
	% appropriate number for your problem.
	%
	% If your problem does not involve a formal proof, you
	% can change the word "proof" in the \begin and \end
	% commands with "solution".
	% -----------------------------------------------------
	
	\begin{problem}{1}
		 Give an efficient dynamic programming algorithm to find the longest palindrome that is a sub-sequence of a given input string. 
	\end{problem}
	
\begin{enumerate}[label=(\alph*)]

\item Characterize longest palindrome.\newline
The longest palindrome will be the longest common sub-sequence that appears the same when read in reverse. The length of a sub-sequence that is a palindrome can be characterized by,

\[ \textup{table}[i,j] = \left\{
\begin{array}{ll}
      1 & \quad i = j \\
      2 & \quad i=j-1 \textup{ \& }  \textup{str}[i] = \textup{str}[j]  \\
      \textup{table}[i+1,j-1] +2  & \quad i<j-1 \textup{ \& }  \textup{str}[i] = \textup{str}[j] \\
      MAX(\textup{table}[i,j-1], \textup{table}[i+1,j])  & \quad i<j-1 \textup{ \& }  \textup{str}[i] \neq \textup{str}[j] \\
\end{array} 
\right. \]

\item Define a recursive solution.\newline
	\begin{algorithm}
		\underline{function rec-Sol}(str[$1..n$],$i$,$j$)\;
		
		\uIf{$i==j$}{
			return $1$\;
		}
			
		\uIf{\textup{str[$i$]}$ ==$ \textup{str[$j$]} }{
			\uIf{$i == j-1$ }{
				return $2$\; 	
			}
			\uElse{
				return rec-Sol(str, $i+1$,$j-1$) $+ 2$
			}			
			
		}
		\Else{
			return $MAX$(rec-Sol(str, $i+1$,$j$) , rec-Sol(str, $i$,$j-1$))\;	
		}			
	\end{algorithm}

\item Compute the length of a longest palindrome sub-sequence.\newline
	\begin{algorithm}
		\underline{function Longest-Palindrome}(str[$1..n$])\;
		Let table[$1..n$,$1..n$] be a two dimensional array\;
		
	
		\For{$r \gets 0 \textbf{\upshape{ to }}  n$}{
			table[$r$][$r$]  $\gets 1$\;
		}
		
		\For{$i \gets 2 \textbf{\upshape{ to }}  n$}{
			\For{$r \gets 0 \textbf{\upshape{ to }}  n-i+1$}{
				$c \gets r+i-1$	\;			
				\uIf{\textup{str[$r$]}$ ==$ \textup{str[$c$]} $\&$ $i==2$ }{
					table[$r$][$c$] $\gets 2$\; 
				}
				\uElseIf{\textup{str[$r$]} $==$ \textup{str[$c$]}}{
					table[$r$][$c$] $\gets$ table[$r+1$][$c-1$] $+2$\; 
				}
				\Else{
					table[$r$][$c$]  $\gets$ $MAX$(table[$r$][$c-1$] , table[$r+1$][$c$])\;	
				}			
			}		
		}
		
		return table[$0$][$n-1$]\;
	\end{algorithm}
	
	\newpage
\item Construct a longest palindrome sub-sequence. (assume that table[$0$][$n-1$] is even)\newline
	\begin{algorithm}
		\underline{function Generate-LPS}(str[$1..n$], table[$1..n$,$1..n$])\;
		Let result be an empty string\;		
		end $\gets$ table[$0$][$n-1$]\;	
		$r \gets 0$\;
		$c \gets n-1$\;	
		
		\While{\textup{end}$\geq 0$ $\&$ $r \leq c$}{
			
				\uIf{ \textup{str[$r$]} $==$ \textup{str[$c$]}}{
					result $\cup$ str[$r$]\;					
					end$--$\;					
					$r++$\;
					$c--$\;
				}   
				\Else{
					\uIf{\textup{table[$r+1$][$c$]} $>$ \textup{table[$r$][$c-1$]}}{
						$r++$\;
					}
					\Else{
						$c--$\;
					}				
				}
			}
		
			$r\gets0$\;
			$ m$ $\gets$ table[$0$][$n-1$]$/2$\;
			$c \gets n - 1$\;
			\While{$c \geq m$}
			{
				result[$r$] $\gets$ result[$c$]\;
				$r++$\;
				$c--$\;
			}
		return result\;
		
	\end{algorithm}
	
	
\item The running time of the algorithm is $\Theta(n^2)$.
\end{enumerate}

\newpage 
\begin{problem}{2}
Java implementation of Longest Common Palindrome Sub-sequence with examples. 
\end{problem}

\lstinputlisting[language=Java]{LCSpalindrome.java}
Output from the code above: 
\lstinputlisting[language={}]{output.txt}

\end{document}
